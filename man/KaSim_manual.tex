\documentclass[11pt]{book}
\frenchspacing
\usepackage{geometry}                % See geometry.pdf to learn the layout options. There are lots.

\usepackage[parfill]{parskip}    % Activate to begin paragraphs with an empty line rather than an indent
\usepackage{graphicx}
\usepackage{amssymb}
\usepackage{epstopdf}
\DeclareGraphicsRule{.tif}{png}{.png}{ 'convert #1 ' dirname #1' /' basename #1 .tif' .png}

\usepackage[T1]{fontenc}
\usepackage{ textcomp }
\usepackage{hyperref}


%% LaTeX - Article customise

%%% PACKAGES
\usepackage{booktabs} % for much better looking tables
\usepackage{array} % for better arrays (eg matrices) in maths
\usepackage{verbatim} % adds environment for commenting out blocks of text & for better verbatim
\usepackage{amsmath}

\usepackage{makeidx}
\usepackage{color}

\usepackage{fancyhdr} %to insert logo on the left head of each page
\pagestyle{fancy}
\begin{titlepage}
\rhead{\includegraphics[width=35mm]{img/KaSim-logo.png}}
\end{titlepage}

\def\KaSim{\textsf{KaSim}}
\def\sep{\hbox{-}}
\def\int{\hbox{\texttt{\~}}}
\def\tcb#1{\textcolor{blue}{\ttt{#1}}}
\def\ttt#1{\texttt{#1}}
\def\var#1{\textquotesingle #1\textquotesingle}

\makeindex

%% END Article customise

%%%% added by Vincent to compensate for the lack of mystyle.sty
\frenchspacing
\def\rar{\rightarrow}
\def\lrar{\leftrightarrow}

\def\ka{\kappa}
\def\ga{\gamma}
\def\bs{\backslash}
\def\noi{\noindent}
\def\ie{ie }
\def\via{via }
\def\set#1{\{#1\}}
\def\ITE#1{\begin{itemize}#1\end{itemize}}
\def\ENU#1{\begin{enumerate}#1\end{enumerate}}
\def\mit#1{{\mathit #1}}
\def\Real{\mathbb R}
\def\Nat{\mathbb N}
\def\dd{-\hspace{0.001cm}-}
\def\imp#1{\emph{#1}\index{#1}}
%%%%
\def\version{3}


\title{KaSim\version~reference manual\\ \small (release 3.5)}
\author{J\'er\^ome Feret and Jean Krivine\thanks{corresponding author: jean.krivine@pps.univ-paris-diderot.fr} \\\url{KappaLanguage.org}}

\date{}       
                                    % Activate to display a given date or no date
\begin{document}
\maketitle

%\begin{center}\includegraphics[width=80mm]{img/wip.jpg}
%\vspace{3cm}
%\\This document is work in progress...
%\end{center}

\tableofcontents
\listoftables

\chapter{Introduction}
\begin{center}\includegraphics[width=9cm]{img/KaSim-logo.png}\end{center}

\section{Preamble}
This manual describes the usage of \KaSim, the latest implementation of Kappa, one member of the growing family of rule-based languages. Rule-based modelling has attracted recent attention in developing biological models that are concise, comprehensible, and easily extensible. 
Although this manual contains a self-contained description of Kappa, it is \emph{not} intended as a tutorial on rule-based modeling.%%Therefore, in the following of this manual some familiarity with Kappa is assumed and 
%
\footnote{For an idea of how Kappa is used in a modeling context, the reader might find the following short note useful \href{http://www.pps.jussieu.fr/~danos/pdf/eov.pdf}{Agile modelling of cellular signalling (SOS'08)}. A longer article, expounding on the causal aspects of modeling, is also available: \href{http://www.pps.jussieu.fr/~danos/pdf/ka-fix.pdf}{Rule-based modelling of cellular signalling (CONCUR'07)}, See also this tutorial: \href{http://www.pps.jussieu.fr/~danos/pdf/mytdg.pdf}{Modelling epigenetic information maintenance: a Kappa tutorial (CAV'09)}. 
}%Further explanations, resources and tools can be found there: \url{http://KappaLanguage.org}. 



\section{The \KaSim~engine}
\KaSim~is an open source stochastic simulator of rule-based models~\cite{DanLan04,Dan_etal07a,Fae_etal05} written in Kappa. Basically \KaSim~takes one or several \hyperref[chap:kappa]{kappa files} as input and generates stochastic trajectories of various observables. \KaSim~implements Danos \textit{et al}'s simulation algorithm~\cite{Dan_etal07b} which adapts Gillespie's algorithm~\cite{Gil76,Gil77} to rule-based models. 

A \emph{simulation event}\index{event} corresponds to the application of a rewriting rule, contained in the kappa files, to the current graph\index{graph} (also called a \emph{mixture}\index{mixture}). The rule is selected according to its \emph{activity}\index{activity}, \ie the number of instances it has in the current mixture\index{mixture}, multiplied by its kinetic rate\index{kinetic rate}, and applied to one of its possible instances in the graph. It results in a new graph together with an updated activity\index{activity} for all rules (see Fig.~\ref{fig:event-loop}). Importantly, the cost of a simulation event is bounded by a constant that is independent of the size of the graph it is applied to~\cite{Dan_etal07b}. 

\begin{figure}[htbp]
\begin{center}
\includegraphics[width=14cm]{img/event-loop.png}
\caption{The event loop}
\label{fig:event-loop}
\end{center}
\end{figure}

Note that \KaSim~is not equipped with a curve visualization tool. However, data outputs are given in a text format that can be displayed using any standard plotting software such as \href{http://www.gnuplot.info/}{gnuplot}.

\section{Support}
\ITE{
\item[-] Kappa language tutorials and downloads: \url{http://kappalanguage.org}
\item[-] Bug reports should be posted on github: \url{https://github.com/jkrivine/KaSim/issues}
\item[-] Questions and answers on the kappa-user mailing list: \url{http://groups.google.com/group/kappa-users}
\item[-] Want to contribute to the project? \ttt{jean.krivine@pps.univ-paris-diderot.fr}
}

\chapter{Installation}\label{chap:install}

\section{Using precompiled binaries}
The easiest way to use \KaSim~is to use pre-compiled versions available in the source code repository (see Section~\ref{sec:sources}), in the \texttt{bin/} folder. Alternatively your can download the binaries at \url{https://github.com/jkrivine/KaSim/tree/master/bin} and then clicking on the \texttt{raw} button after selecting the executable that corresponds to your architecture. Download the version that corresponds to your operating system (Windows, Linux or Mac OSX) and rename the downloaded file into \KaSim. Note that on Mac OSX or linux it might be necessary to give executable permissions to \KaSim. This can be done using the shell command:  
\ttt{chmod u+x KaSim}. 

To test whether your program does work, simply type \ttt{./KaSim \dd version} on a terminal, from the directory that contains the binaries. If the version is displayed it means that the binaries are indeed compatible with your OS. Otherwise you may need to compile \KaSim~from the sources (see next Section).

\section{Obtaining the sources}\label{sec:sources}
To obtain \KaSim~you can either use pre-compiled binaries (see previous section) or compile the sources for your architecture. To do so, download the source code from \url{https://github.com/jkrivine/KaSim} and make sure you have a recent ocaml compiler installed (\KaSim~requires Ocaml 3.12 to compile). From a terminal window type \ttt{ocamlopt.opt -v}. If nothing appears then you need to install Ocaml Native compiler that can be downloaded from \url{http://caml.inria.fr/download.en.html}. 

\section{Compilation}
Once Ocaml is safely installed, untar \KaSim~archive and compile following these few steps:
\ITE{
\item[\$]\ttt{tar xzvf kasim.tar.gz -d Kappa}
\item[\$]\ttt{cd Kappa}
\item[\$]\ttt{make}
}
At the end of these steps you should see, in the \ttt{Kappa} directory, an executable file named KaSim.  In order to check the compilation went fine, simply type $.\bs$\ttt{KaSim -\,-version}. If the ocaml native compiler \ttt{ocamlopt.opt} is not in the path of your system, you may set the variable \ttt{OCAMLBINPATH} to point to the location of the compiler by editing the corresponding line in the Makefile.

\chapter{The command line}\label{chap:cl}

\section{General usage}
From a terminal window, \KaSim~can be invoked by typing 
\ITE{
\item[\$] \ttt{KaSim -i file\_1 ... -i file\_n [option] }
}
where \ttt{file\_i} are the input kappa files containing the rules, initial conditions and observables, see Chapter~\ref{chap:kappa} below. Tables~\ref{tab:options} and \ref{tab:add-options} summarize all the options that can be given to the simulator. Basically, one specifies an upper bound either in bio time (arbitrary time unit), or in number of events. Note that bio-time is computed using Gillespie\textquotesingle s formula for time advance (see Fig.~\ref{fig:event-loop}) and should not be confused with CPU-time (it\textquotesingle s not even proportional). In doubt, we recommend using a bound in number of events since the cost of one event application is bounded (in CPU time) by a constant, so the simulation time of $n$ events is roughly $k$ times faster than a simulation of $k*n$ events.

\section{Main options}

Table~\ref{tab:options} summarizes the main options that are accessible through the command line. Options that expects an argument are preceded by a single dash, options that do not need any argument start with a double dash. Note that the option \ttt{-p} specifies the number of points that one wishes to have in the final plot. The interval at which these points will be taken is then computed using the simulation limit defined by the user using option \ttt{-t} or \ttt{-e}. For instance requiring a simulation of 100 points during 10 time units will result in a simulation where observables are recorded every 0.1 time units, while requiring 100 points for 1000 events will result in observable being written every 10 events (be careful that the event density, \ie the number of events per time unit, might vary during a simulation, and thus the two sampling methods can lead to very different repartitions of the moments at which observables are recorded even if they specify the same number of observations). 

\begin{table}[htdp]
\caption{Command line: main options}
\begin{center}
\begin{tabular}{|l|l|}
\hline 
Argument & Description \\ \hline
\ttt{-e $e_\mit{max}$} & Terminates simulation after $e_\mit{max}\geq 0$ events\\
\ttt{-t $t_\mit{max}$} & Terminates simulation after $t_\mit{max}\geq 0.0$ time units\\
\ttt{-p $n$} & Produces a data file\index{data file} (default: \ttt{data.out}) with $n\geq 0$ data points\\
\ttt{-o} \textit{file} & Set the name of data file to \textit{file}\\ 
\ttt{-d} \textit{dir} & Output any produced file to the directory \textit{dir}\\
\hline
\end{tabular}
\end{center}
\label{tab:options}
\end{table}%

\section{Advanced options}

Table~\ref{tab:add-options} summarizes the advanced options that are accessible through the command line.

\begin{table}[htdp]
\caption{Command line: advanced options}
\begin{center}
\begin{tabular}{|l|l|}
\hline 
Argument & Description \\ \hline
\ttt{\small -seed $n$} &\small Seeds the pseudo-random number generator $n>0$ \\
\ttt{\small \dd implicit-signature} &\small Automatically deduce agent signature\index{agent signature}s (see Chapter~\ref{chap:advanced})\\
\ttt{\small -make-sim} \textit{sim\_file} &\small makes a simulation package out of the input kappa files \\
\ttt{\small -load-sim} \textit{sim\_file} &\small use simulation package \textit{sim\_file} as input\\
\ttt{\small \dd gluttony} &\small simulation mode that is memory intensive \\ &\small but that speeds up simulation time\\

\hline
\end{tabular}
\end{center}
\label{tab:add-options}
\end{table}%
%
\section*{Example}
The command 
\ITE{
\item[\$] \ttt{KaSim -i model.ka -e 1000000 -p 1000 -o model.out}  
}
will generate a file \ttt{model.out} containing the trajectories of the observables defined in the kappa file \ttt{model.ka}. The file \ttt{model.out} will contain 1000 data points (\ie in this case, a measure will be taken every 1000 events). The command
\ITE{
\item[\$] \ttt{KaSim -i init.ka -i rules.ka -i obs.ka -i mod.ka -t 1.5 -p 1000}  
}
will generate a file \ttt{data.out} (default name) containing 1000 data points of a simulation of 1.5 seconds (arbitrary time units) of the model. Note that the input kappa file is split in 4 files containing, for instance, the initial conditions, \ttt{init.ka}, the rule set, \ttt{rules.ka}, the observables, \ttt{obs.ka}, and the perturbations, \ttt{pert.ka} (see Chapter~\ref{chap:kappa} for details). The order in which the files are given does not matter.

\chapter{The kappa file}\label{chap:kappa}

\section{General remarks}
The \emph{Kappa File}\index{kappa file} (KF) is the formal representation of your model. We use KF\index{kappa file} to denote the union of the files that are given as input to \KaSim~(argument \ttt{-i}). Each line of the KF\index{kappa file} is interpreted by \KaSim~as a \emph{declaration}\index{declaration}. If the line is ended by the escape character~\textquotesingle \ttt{$\bs$}\textquotesingle ~the continuation of the declaration\index{declaration} is parsed onto the next line. Declarations can be: agent and token \emph{signatures} (Sec.~\ref{sec:sig}), \emph{rules}\index{rule} (Sec.~\ref{sec:rules}), \emph{variables}\index{variable} (Sec.~\ref{sec:var}), \emph{initial conditions}\index{initial condition} (Sec.~\ref{sec:init}), \emph{perturbations}\index{perturbation} (Sec.~\ref{sec:mod}) and \emph{parameter configurations} (Sec.~\ref{sec:param}). The KF\index{kappa file}'s structure is quite flexible and can be divided in any number of sub-files in which the order of declaration\index{declaration}s does not matter (to the exception of variable declaration\index{declaration}s, see Section~\ref{sec:var} for details). Comments\index{comments} can be used by inserting the marker \ttt{\#} that tells \KaSim~to ignore the rest of the line. 

\section{Agent and token signature\index{agent signature}s}\label{sec:sig}
%
In Kappa there are two entities that can be used for representing biological elements: \imp{agents} and \imp{tokens}.  Agents are used to represent complex molecules that may bind to other molecules on specific sites. Tokens are used to represent atomic particles such as calcium ions, atp etc. Tokens cannot bind to each others, they can only appear or disappear. In a given model, agents always have a discrete number of instances while tokens may have a continuous concentration.

In order to use agents or tokens in a model, one needs to declare them first. \emph{Agent signatures}\index{agent signature} constitute a form of typing information about the agents that are used in the model. It contains information about the name and number of interaction sites the agent has, and about their possible internal states. A signature\index{agent signature} is declared in the KF\index{kappa file} by the following line:
\ITE{
\item[] \ttt{\%agent: } \textit{signature\_expression}
}
according to the grammar given Table~\ref{tab:sig} where terminal symbol are denoted in (blue) typed font. Symbol \ttt{Id} can be any string generated by regular expression $[a\sep z\ A\sep Z][a\sep z\ A \sep Z\ 0\sep 9\ \_ \ -\ +]^*$. Terminal symbol $\varepsilon$ stands for the empty symbol.
%%
\begin{table}[htbp]
  \centering
  \caption{Agent signature\index{agent signature} expression}
  \begin{tabular}{@{} lcl @{}}
    \textit{signature\_expression} & ::= & \tcb{Id}\tcb{(}\textit{sig}\tcb{)} \\ 
    \textit{sig} & ::= & \tcb{Id}~\textit{internal\_state\_list}\tcb{,}\ \textit{sig} $\mid\varepsilon$ \\ 
    \textit{internal\_state\_list} & ::= & \tcb{\int Id}~\textit{internal\_state\_list} $\mid\varepsilon$
    \end{tabular}
  \label{tab:sig}
\end{table}
%%

For instance the line:
\ITE{
\item[] \ttt{\%agent: A(x,y\int u\int p,z\int 0\int 1\int 2) \# Signature of agent A} 
}
will declare an agent \ttt{A} with 3 \emph{(interaction) sites} \ttt{x,y} and \ttt{z} with the site \ttt{y} possessing two \emph{internal states}\index{internal state} \ttt{u} and \ttt{p} (for instance for the unphosphorylated and phosphorylated forms of \ttt{y}) and the site \ttt{z} having possibly 3 states respectively 0, 1 and 2. Note that internal states\index{internal state} values are treated as untyped symbols by \KaSim, so choosing a character or an integer as internal state is purely matter of convention. 

Token signature\index{agent signature}s are declared using a statement of the form:
\ITE{
\item[] \ttt{\%token: ca+ \# Signature of calcium token} 
}

\section{Rules}\label{sec:rules}

Once agents are declared, one may add to the KF\index{kappa file} the rules\index{rule} that describe their dynamics through time. A \imp{pure rule} looks like: 
\ITE{
\item[] \ttt{\textquotesingle my rule\textquotesingle } {\it kappa\_expression} $\rar$ {\it kappa\_expression} \ttt{@} {\it rate}  
}
where \ttt{\textquotesingle my rule\textquotesingle } can be any name that will refer to the subsequent rule that can be decomposed into a \emph{left hand side}\index{left hand side} (LHS) and a \emph{right hand side}\index{right hand side} (RHS) kappa expressions\index{kappa expressions} together with a \emph{kinetic rate expression} \index{kinetic rate}. One may also declare a \imp{bi-directional rule} using the convention:
\ITE{
\item[] \ttt{\textquotesingle bi-rule\textquotesingle } {\it kappa\_expression} $\lrar$ {\it kappa\_expression} \ttt{@} {\it rate$^+$,rate$^-$}  
}
Note that the above declaration\index{declaration} corresponds to writing, in addition of \ttt{\var{my-rule}}, a backward rule named \ttt{\var{my rule\_op}} which swaps left hand side and right hand side, and with rate \textit{rate$^-$}.


Kappa and rate expressions are generated by the grammar given in Table~\ref{tab:patterns}.

If pure rules induce reactions between agents, it is possible to mix agents and tokens in \imp{hybrid rules} (which may also be bi-directional) which look like:
 \ITE{
\item[] {\it kappa\_expression} | {\it token\_expression} $\rar$ {\it kappa\_expression}  | {\it token\_expression}\ttt{@} {\it rate}  
}
Token expressions are generated by the grammar given Table~\ref{tab:patterns} (algebraic expressions\index{algebraic expression} are given Table~\ref{tab:alg}, but can be replaced for now as any real number).
%%
\begin{table}[htbp]
  \centering
  \caption{Kappa expressions}
  \begin{tabular}{@{} lcl @{}}
    \textit{kappa\_expression} & ::= & \textit{agent\_expression} \tcb{,} \textit{kappa\_expression} $\mid\varepsilon$ \\ 
    \textit{agent\_expression} & ::= &  \tcb{Id}\tcb{(}\textit{interface}\tcb{)} \\
    \textit{interface} &::=& $\varepsilon\mid$ \tcb{Id} \textit{internal\_state link\_state}\\
    \textit{internal\_state} &::=& $\varepsilon$ | \tcb{\int Id} \\
    \textit{link\_state} &::=& $\varepsilon$ | \tcb{!n} | \tcb{!\_} | \tcb{?}  \\\\
   \textit{token\_expression} &::=&  \textit{algebraic\_expression}\tcb{:}\textit{token\_name} \\
	&&|  \textit{token\_expression} \tcb{+}\ \textit{token\_expression} \\\\
\textit{token\_name} &::=& \tcb{Id}\\\\

    \textit{rate\_expression} &::=& \textit{algebraic\_expression} \\
&&| \textit{algebraic\_expression} \tcb{(}\textit{algebraic\_expression}\tcb{)} 
    \end{tabular}
  \label{tab:patterns}
\end{table}
%%
\subsection{Pure rules}\index{pure rule}

\subsubsection*{A simple rule}
With the signature\index{agent signature} of \ttt{A} defined in the previous section, the line
\ITE{
\item[] \ttt{\textquotesingle A dimerization\textquotesingle~  A(x),A(y\int p)} $\rar$ \ttt{A(x!1),A(y\int p!1) @ $\ga$} 
}
denotes a dimerization rule between two instances of agent \ttt{A} provided the second is phosphorylated (say that is here the meaning of \ttt{p}) on site \ttt{y}. Note that the bond between both \ttt{A}s is denoted by the identifier \ttt{!1} which uses an arbitrary integer (\ttt{!0} would denote the same bond). In Kappa, a bond may connect exactly 2 sites so any occurrence of a bond identifier \ttt{!}$n$ has to be paired with exactly one other sibling in the expression. Note also the fact that site \ttt{z} of \ttt{A} is not mentioned in the expression which means that it has no influence on the triggering of this rule. This is the \emph{don't care don't write convention} (DCDW) \index{don't care don't write} that plays a key role in resisting combinatorial explosion when writing models.

\subsubsection*{Adding and deleting agents}
Sticking with \ttt{A}'s signature\index{agent signature}, the rule
\ITE{
\item[] \ttt{\var{budding A}~A(z)} $\rar$ \ttt{A(z!1),A(x!1) @ $\ga$ } 
}
indicates that an agent \ttt{A} free on site \ttt{z}, no matter what its internal state is, may beget a new copy of \ttt{A} bound to it \via site \ttt{x}. Note that in the RHS, agent \ttt{A}' s interface is not completely described. Following the DCDW convention\index{don't care don't write}, \KaSim~will then assume that the sites that are not mentioned are created in the \emph{default state}\index{default state}, \ie they appear free of any bond and their internal state (if any) is the first of the list shown in the signature\index{agent signature} (here state \ttt{u} for \ttt{y} and \ttt{0} for \ttt{z}).

Importantly, \KaSim~respects the \emph{longest prefix convention}\index{longest prefix convention} to determine which agent in the RHS stems from an agent in the LHS. 
In a word, from a rule of the form $a_1,\dots,a_n\rar b_1,\dots,b_k$, with $a_i$s and $b_j$s being agents, one computes the biggest indices $i\leq n$ such that the agents $a_1,\dots,a_i$ are pairwise consistent with $b_1,\dots,b_i$, \ie the $a_j$s and $b_j$s have the same name and the same number of sites. In which case we say that the for all $j\leq i$, $a_j$ is \emph{preserved} by the transition and for all $j>i$, $a_j$ is \emph{deleted} by the transition and $b_j$ is \emph{created} by the transition. This convention allows us to write a deletion rule as:
\ITE{
\item[] \ttt{\textquotesingle deleting A\textquotesingle\  A(x!1),A(z!1)} $\rar$ \ttt{A(x) @ $\ga$} 
}
which will remove the \ttt{A} agent in the mixture\index{mixture} that will match the second occurrence of \ttt{A} in this rule.

\subsubsection*{Side effects}\label{sec:side-effect}

It may happen that the application of a rule has some \emph{side effects}\index{side effect} on agents that are not mentioned explicitly in the rule. Consider for instance the previous rule:
\ITE{
\item[] \ttt{\textquotesingle deleting A\textquotesingle  A(x!1),A(z!1)} $\rar$ \ttt{A(x) @ $\ga$} 
}
The \ttt{A} in the graph that is matched to the second occurrence of \ttt{A} in the LHS will be deleted by the rule. As a consequence all its sites will disappear together with the bonds that were pointing to them. For instance, when applied to the graph 
\ITE{
\item[] $G=$\ttt{A(x!1,y\int p,z\int2),A(x!2,y\int u,z\int0!1),C(t!2)}
} 
the above rule will result in a new graph $G' =\ttt{A(x!1,y\int p,z\int2),C(t)}$ where the site \ttt{t} of \ttt{C} is now free as side effect.

\emph{Wildcard} symbols for link state \ttt{?} (for bound or not), \ttt{!\_} (for bound to someone), may also induce side effects when they are not preserved in the RHS of a rule, as in
\ITE{
\item[] \ttt{\textquotesingle Disconnect A\textquotesingle ~ A(x!\_)} $\rar$ \ttt{A(x) @ $\ga$} 
}
or 
\ITE{
\item[] \ttt{\textquotesingle Force bind A\textquotesingle  ~ A(x?)} $\rar$ \ttt{A(x!1),C(t!1) @ $\ga$} 
}
%%
Both these rule will cause \KaSim~to raise a warning at rule compilation time.

\subsection{Hybrid rules}\label{sec:hybrid}

Using KaSim \imp{hybrid rules}, one may declare that an action has effects on the concentration of some particles of the system. For instance a rule may consume atp, calcium ions etc. It would be a waste of memory and time to use discrete agents to represent such particles. Instead one may declare tokens using declarations\index{declaration} of the form:
\ITE{
\item[] \ttt{\%token: atp} 
\item[] \ttt{\%token: adp} 
}
One may then use these tokens in conjunction with a classical rule using the hybrid format:
\ITE{
\item[] \ttt{\var{hybrid rule}~ S(x\int u!1),K(y!1)\ |\ 0.1:atp $\rar$ S(x\int p),K(y)\ |\ 0.1:adp\ @ \var{k}} 
}
When applied, the above rule will consume 0.1 \ttt{atp} token and produce 0.1 \ttt{adp} token. Note that as specified by the grammar given Table~\ref{tab:patterns}, the number of consumed (and produced) tokens can be given by a sum of the form:
\ITE{
\item[] \ttt{\textit{lhs} | a$_1$:t$_1$ + ... + a$_n$:t$_n$ $\rar$ \textit{rhs} | a$_1'$:t$_1'$ + ... + a$_k'$:t$_k'$ @ r} 
}
where each $a_i,a_i'$ is an arbitrary algebraic expression\index{algebraic expression} (see Table~\ref{tab:alg}) and each $t_i,t_i'$ is a declared token. In the above hybrid rule, calling $n_i,n_i'$  the evaluation of $a_i$ and $a_i'$, the concentration of token $t_i$ will decrease from $n_i$ and the concentration of token $t_i'$ will increase from $n_i'$. Importantly the activity\index{activity} of a hybrid rule like the above one is still 
defined by $\ttt{|\textit{lhs}|*r}$, where $\ttt{|\textit{lhs}|}$ is the number of embeddings of the lhs of the rule in the mixture, and does not take into account the concentration of the tokens it mentions. As we will see in the next section, it is however possible to make its rate explicitly depend on the concentrations of the tokens using a \imp{variable} rate.

\subsection{Rates}\label{sec:kinetics}
As said earlier, Kappa rules are equipped with one or two \emph{kinetic rate(s)}\index{kinetic rate}. A rate is a real number, or an algebraic expression\index{algebraic expression} evaluated as such, called the \emph{individual-based or stochastic rate constant}\index{kinetic rate!stochastic rate constant}, it is the rate at which the corresponding rule is applied per instance of the rule. Its dimension is the inverse of a time $[T^{-1}]$.

The stochastic rate is related to the \emph{concentration-based rate constant}\index{kinetic rate!deterministic rate constant} $k$ of the rule of interest by the following relation:
%corrected by the volume $V$ in which the model is considered. 
%
%%We often want to express $\ga$ in \textquotesingle \textquotesingle per molecule\textquotesingle \textquotesingle  rather than \textquotesingle \textquotesingle per mole\textquotesingle \textquotesingle . 
%%This only changes the numerical value, not the dimension (both are numbers of molecules): 
%
%%The relation between both rate constant is:
\begin{equation}\label{stochrate}
k = \ga{({\cal A}\; V)^{(a-1)}} %\qquad [\mathit{molecule}^{-1} \mathit{s}^{-1}],
\end{equation}
where 
$V$ is the volume where the model is considered,
${\cal A}= 6.022\cdot 10^{23}$ is Avogadro' s number, %number of $H$ atoms needed to make up for a gram of $H$, 
$a\geq0$ is the arity of the rule (\ie 2 for a bimolecular rule).

In a modeling context, the constant $k$ is typically expressed using \emph{molars} $M:=\mit{moles}\,l^{-1}$ (or variants thereof such
as $\mu M$, $nM$), and seconds or minutes. If we choose molars and seconds, $k$' s unit is $\mathit{M}^{1-a}\mathit{s}^{-1}$, as 
follows from the relation above.

Concentration-based rates are usually favored for measurements and/or deterministic models, so it is useful to know how to convert them into
individual-based ones used by KaSim. 
%We refer the reader to \url{http://www.KappaLanguage.org} for a tutorial on kinetic rates. 
Here are typical volumes used in modeling:
\begin{itemize}
\item
Mammalian cell: $V=2.25\; 10^{-12}l$  ($1l=10^{-3}m^3$), and ${\cal A}V=1.35\; 10^{12}$. 

A concentration of $1M$ in a mammalian cell volume corresponds to $1.35\; 10^{12}$ molecules; $1nM\approx 1350$ molecules per cell.
\item
Yeast cell (haploid): $V=4\; 10^{-14}l$, and ${\cal A}V=2.4\; 10^{10}$. 

A concentration of $1M$ in a yeast cell volume corresponds to $2.4\; 10^{10}$ molecules; $1nM\approx 24$ molecules per cell. The volume is doubled in a diploid cell.
\item
E.\@ Coli cell: $V=10^{-15}l$, and ${\cal A}V=10^{8}$. 

A concentration of $1M$ in a yeast cell volume corresponds to $10^{8}$ molecules; $10nM\approx 1$ molecule per cell.
\end{itemize}

The table below lists typical ranges for deterministic rate constants and 
their stochastic counterparts assuming a mammalian cell volume.

\begin{table}[htbp]
\centering
\caption{Example of kinetic rates.}
\begin{tabular}{@{} lllr @{} }
\toprule
process & $k$ & $\gamma$ %& stoch. dimension
\\
\midrule
general binding & $10^{7}-10^{9}$ & $10^{-5}-10^{-3}$ %& $s^{-1}$ %$molecule^{-1} s^{-1}$
\\
general unbinding &  $10^{-3} - 10^{-1}$ & $10^{-3}-10^{-1}$ %& $s^{-1}$
\\
dephosphorylation & 1 & 1 %& $s^{-1}$
\\
phosphorylation & 0.1 & 0.1 %& $s^{-1}$
\\
receptor dimerization & $2\; 10^{6}$ & $1.6 \;10^{-6}$ %& $s^{-1}$%$molecule^{-1} s^{-1}$
\\
receptor dissociation & $1.6\; 10^{-1}$ & $1.6\; 10^{-1}$ %& $s^{-1}$
\\
\bottomrule
\end{tabular}
\end{table}

\subsection{Ambiguous molecularity}\label{sec:ambiguous}

It is considered malpractice to use a Kappa rule of the form \ttt{A(x),B(y)$\rar \dots$ @ $\ga$} in a model where this rule could be applied in a context where \ttt{A} and \ttt{B} are sometimes already connected and sometimes disconnected. Indeed, this would lead to an inconsistency in the definition of the kinetic rate $\ga$ which should have a volume dependency in the former case and no volume dependency in the latter (see Section~\ref{sec:kinetics}). 

This sort of ambiguity should be resolved, if possible, by refining the ambiguous rule into cases that are either exclusively unary or binary. Each refinement having a kinetic rate that is consistent with its molecularity. Note that in practice, for models with a large number of agents, it is sufficient to assume that the rule \ttt{A(x),B(y)$\rar \dots$ @ $\ga$} will have only binary instances. In this case it suffices to consider the approximate model:
\begin{itemize}
\item[]\ttt{\textquotesingle assumed binary AB\textquotesingle ~  A(x),B(y)$\rar \dots$ @ $\ga_2$}
\item[]\ttt{\textquotesingle unary AB\textquotesingle  ~ A(x,c!1),C(a!1,b!2),B(y,c!2)$\rar \dots$ @ $k_1$}
\end{itemize}

There are however systems where even enumerating unary cases becomes impossible. As an alternative, one should use the kappa notation for ambiguous rules: 
\ITE{
\item[] \ttt{\textquotesingle my rule\textquotesingle } {\it kappa\_expression} $\rar$ {\it kappa\_expression} \ttt{@} $\ga_2(k_1)$  
}
which will tell \KaSim~to apply the above rule with a rate $\ga_2$ for binary instances and a rate $k_1$ for unary instances. The obtained model will behave exactly as a model in which the ambiguous rule has been replaced by unambiguous refinements. However the usage of such rule might \emph{slowdown simulation in a significant manner} depending on various parameters (such as the presence of large polymers in the model). We give below an example of a model utilizing binary/unary rates for rules\footnote{This model is available in the source repository \ttt{models/poly.ka}.}.

{\tt\footnotesize
\begin{enumerate}
\item \%agent: A(b,c)
\item \%agent: B(a,c)
\item \%agent: C(b,a)
\item \#\#
\item \%var: \var{V}~  1
\item \%var: \var{k1}~  [inf]
\item \%var: \var{k2}~ 1.0E-4/\var{V} 
\item \%var: \var{k\_off} ~ 0.1
\item \#\#
\item \var{a.b}~ A(b),B(a) -> A(b!1),B(a!1) @ \var{k2}~(\var{k1})
\item \var{a.c}~ A(c),C(a) -> A(c!1),C(a!1) @ \var{k2}~(\var{k1})
\item \var{b.c}~ B(c),C(b) -> B(c!1),C(b!1) @ \var{k2}~(\var{k1})
\item \#\#
\item \var{a..b}~A(b!a.B) -> A(b) @ \var{k\_off} 
\item \var{a..c}~A(c!a.C) -> A(c) @ \var{k\_off}
\item \var{b..c}~B(c!b.C) -> B(c) @ \var{k\_off} 
\item \#\#
\item \%var: \var{n}~  1000
\item \#\#
\item \%init: \var{n}~  A(),B(),C()
\item \%mod: [E] > 10000 do \$STOP 
\item \%def: \var{\var{dotSnapshots}} \var{\var{true}}
\end{enumerate}
}

Notice at lines 10-12 the use of binary/unary notation for rules. As a result binding between between freely floating agents will occur at rate \ttt{\textquotesingle k2\textquotesingle } while binding between agents that are part of the same complex will occur at rate \ttt{\textquotesingle k1\textquotesingle }. Line 21 contains a \emph{perturbation} that requires \KaSim~to stop the simulation after 10,000 events and output the list of molecular species present in the final mixture\index{mixture} as a dot file (see Section~\ref{sec:mod}) and that we give Figure~\ref{fig:species}.

\begin{figure}[htbp]
\begin{center}
\includegraphics[width=10cm]{img/poly.pdf}
\caption{Final mixture\index{mixture} obtained after 10,000 events of simulation of the \ttt{poly.ka} model. The infinite rate for cycle closure allows one to obtain a large number of triangles.}
\label{fig:species}
\end{center}
\end{figure}


\section{Variables}\label{sec:var}

In the KF\index{kappa file} it is also possible to declare \emph{variables}\index{variable} with the declaration:
\ITE{
\item[] \ttt{\%var: \var{var\_name}~} (\textit{algebraic\_expression} or \textit{kappa\_expression})
}
where \ttt{var\_name} can be any string and \textit{algebraic\_expression} are defined Table~\ref{tab:alg} (available symbols for variable, constants and operators are given Table~\ref{tab:operators}).

%%
\begin{table}[htbp]
  \centering
  \caption{Algebraic expressions}
  \begin{tabular}{@{} lcl @{}}
    \textit{algebraic\_expression} & ::= & \tcb{$x\in\Real$} $\mid$ \tcb{variable} \\
     &  & $\mid$ \textit{algebraic\_expression} \tcb{binary\_op} \textit{algebraic\_expression}\\
     &  & $\mid$ \tcb{unary\_op} \tcb{(}\textit{algebraic\_expression}\tcb{)} \\ 
    \end{tabular}
  \label{tab:alg}
\end{table}
%%



\begin{table}[htbp]
\centering
\caption{Symbol usable in algebraic expressions\index{algebraic expression}}
\begin{tabular}{@{} l|l @{} }
\toprule
\ttt{\tcb{variable}} & Interpretation \\
\midrule
\ttt{[E]} & the total number of simulation events\index{event}\\
& since the beginning of the simulation \\
\ttt{[E+]} & the total number of productive events\index{event} \\
\ttt{[E-]} & the total number of null events\index{null event}\\
\ttt{[Emax]} & the max (productive) event limit as set by the option \ttt{-e}. \\ &Note that if unset \ttt{Emax}$=\infty$\\
\ttt{[T]} & the bio-time of the simulation \\
\ttt{[Tsim]} & the cpu-time since the beginning of the simulation \\
\ttt{[Tmax]} & the max (bio)-time limit as set by the option \ttt{-t}. \\ &Note that if unset \ttt{Tmax}$=\infty$\\

\ttt{\var{v}} & the value of variable \ttt{\var{v}} (declared using the \ttt{\%var:} statement) \\
\ttt{|t|} & the concentration of token \ttt{t}\\
\ttt{[inf]} & symbol for $\infty$ \\\\
\toprule
\ttt{\tcb{unary/binary\_op}} & Interpretation \\
\midrule
\ttt{[}$f$\texttt{]} & the intuitive mathematical function or constant \\ &associated to $f\in\set{\hbox{\ttt{log,sin,cos,tan,sqrt,pi}}}$ \\
\ttt{[mod]} & the \emph{modulo} operator (infix notation)\\
\ttt{[max]} & the \emph{maximum} between two values \\
\ttt{[min]} & the \emph{minimum} between two values \\
\ttt{[exp]} & the exponentiation operation $x\mapsto e^x$ \\
\ttt{[int]} & the floor function $x\in\Real\mapsto \lfloor x\rfloor\in\Nat^+$ \\
\ttt{+,-,*,/,\^} & the corresponding mathematical operators (infix notation)\\
\bottomrule
\end{tabular}
\label{tab:operators}
\end{table}
%%

\noi For instance the declaration\index{declaration}s
\ITE{
\item[] \ttt{\%var: \var{homodimer}~A(x!1),A(x!1)}
\item[] \ttt{\%var: \var{aa}~\var{homodimer}/2}
}
define 2 variables, the first one tracking the number of embeddings of \ttt{A(x!1),A(x!1)} in the graph over time, while the second divides this value by 2: the number of automorphisms in \ttt{A(x!1),A(x!1)}. Note that variables that are used in the expression of another variable must be declared beforehand. 
\vskip 0.25cm
\noi It is also possible to use algebraic expressions\index{algebraic expression} as kinetic rates\index{kinetic rate} as in
\ITE{
\item[] \ttt{\%var: \var{k\_on}~1.0E-6 \# per molecule per second}
\item[] \ttt{\var{ab}~A(x),A(x) -> A(x!1),A(x!1) @ \var{k\_on}/2}
}
%%
\KaSim~may output values of variables in the data file\index{data file} (see option \ttt{-p} in Chapter~\ref{chap:cl}) using plot do:
\ITE{
\item[] \ttt{\%plot: \textquotesingle var\_name\textquotesingle }
}
One may use the shortcut: 
\ITE{
\item[] \ttt{\%obs: \textquotesingle var\_name\textquotesingle  } \textit{algebraic\_expression} or \textit{kappa\_expression}
}
to declare a variable and at the same time require it to be outputted in the data file\index{data file}. 

\section{Initial conditions}\label{sec:init}\index{initial condition}

The initial mixture\index{mixture} to which rules in the KF\index{kappa file} will be applied are declared as
\ITE{
\item[] \ttt{\%init: } \textit{algebraic\_expression} \textit{kappa\_expression}
}
or
\ITE{
\item[] \ttt{\%init: }  \textit{token} \ttt{<-} \textit{algebraic\_expression}
}

%%
where \textit{algebraic\_expression} is evaluated before initialization of the simulation (hence all token and kappa expression values in the expression are evaluated to 0). This will add to the initial state of the model \textit{mult} copies of the graph described by the kappa expression. Again the DCDW convention\index{don't care don't write} allows us not to write the complete interface of added agents (the remaining sites will be completed according to the agent's signature\index{agent signature}). For instance:
\ITE{
\item[] \ttt{\%var: \textquotesingle n\textquotesingle~1000}
\item[] \ttt{\%init: \textquotesingle n\textquotesingle~(A(),A(y\int p))}
\item[] \ttt{\%init: ca2+ <- 0.39 \#mM}
}
will add 1000 instances of \ttt{A} in its default state \ttt{A(x,y\int u,z\int 0)}, 1000 instances of \ttt{A} in state \ttt{A(x,y\int p,z\int 0)} and a concentration of 0.39 mM of calcium ions. Recall that the concentration of calcium can be observed during simulation using \ttt{|ca2+|}.
As any other declaration\index{declaration}, \ttt{\%init} can be used multiple times, and agents will add up to the initial state.

\chapter{A simple model}\label{chap:abc}

We describe below the content of a simple Kappa model and give examples of some typical run\footnote{The corresponding kappa file is included in the distribution of \KaSim, in the directory \ttt{models/}}.

\section{ABC.ka}
{\footnotesize\ttt{
\ENU{
\item \tcb{\#\#\#\# Signatures}
\item \%agent: A(x,c) \tcb{\# Declaration of agent A}
\item \%agent: B(x) \tcb{\# Declaration of B}
\item \%agent: C(x1\int u\int p,x2\int u\int p) \tcb{\# Declaration of C with 2 modifiable sites}
\item \tcb{\#\#\#\# Rules}
\item \textquotesingle a.b\textquotesingle  A(x),B(x) -> A(x!1),B(x!1) @ \textquotesingle on\_rate\textquotesingle  \tcb{\#A binds B} 
\item \textquotesingle a..b\textquotesingle  A(x!1),B(x!1) -> A(x),B(x) @ \textquotesingle off\_rate\textquotesingle  \tcb{\#AB dissociation} 
\item \textquotesingle ab.c\textquotesingle  A(x!\_,c),C(x1\int u) ->A(x!\_,c!2),C(x1\int u!2)  @ \textquotesingle on\_rate\textquotesingle  \tcb{\#AB binds C} 
\item \textquotesingle mod x1\textquotesingle  C(x1\int u!1),A(c!1) ->C(x1\int p),A(c)  @ \textquotesingle mod\_rate\textquotesingle  \tcb{\#ABC modifies x1} 
\item \textquotesingle a.c\textquotesingle  A(x,c),C(x1\int p,x2\int u) -> A(x,c!1),C(x1\int p,x2\int u!1) @ \textquotesingle on\_rate\textquotesingle  \tcb{\#A binds C on x2}
\item \textquotesingle mod x2\textquotesingle  A(x,c!1),C(x1\int p,x2\int u!1) -> A(x,c),C(x1\int p,x2\int p) @ \textquotesingle mod\_rate\textquotesingle  \tcb{\#A modifies x2} 
\item \tcb{\#\#\#\# Variables}
\item \%var: \textquotesingle on\_rate\textquotesingle  1.0E-4 \tcb{\# per molecule per second}
\item \%var: \textquotesingle off\_rate\textquotesingle  0.1 \tcb{\# per second}
\item \%var: \textquotesingle mod\_rate\textquotesingle  1 \tcb{\# per second}
\item \%obs: \textquotesingle AB\textquotesingle  A(x!x.B)
\item \%obs: \textquotesingle Cuu\textquotesingle  C(x1\int u,x2\int u)
\item \%obs: \textquotesingle Cpu\textquotesingle  C(x1\int p,x2\int u)
\item \%obs: \textquotesingle Cpp\textquotesingle  C(x1\int p,x2\int p)
\item \tcb{\#\#\#\# Initial conditions}
\item \%init: 1000 A(),B()
\item \%init: 10000 C()
}
}}
%

Line 1-4 of this KF\index{kappa file} contains signature\index{agent signature} declarations\index{signature}\index{declaration}. Agents of type \ttt{C} have 2 sites \ttt{x1} and \ttt{x2} whose internal state may be \ttt{u}(nphosphorylated) or \ttt{p}(hosphorylated). Recall that the default state of these sites is \ttt{u} (the first one). Line 8, rule \ttt{\textquotesingle ab.c\textquotesingle } binds an \ttt{A} connected to someone on site \ttt{x} (link type \ttt{!\_}) to a \ttt{C}. Note that the only rule that binds an agent to \ttt{x} of \ttt{A} is \ttt{\textquotesingle a.b\textquotesingle } at line 6. Hence the use of \ttt{!\_} is a commodity and the rule could be replaced by 
\ITE{
\item[]\ttt{\var{alt\_ab.c} A(x!1,c),B(x!1),C(x1\int u) $\rar\dots$}
}
There are two main points to notice about this model: \ttt{A} can modify both sites of \ttt{C} once it is bound to them. However, only an \ttt{A} bound to a \ttt{B} can connect on \ttt{x1} and only a free \ttt{A} can connect on \ttt{x2}. Note also that \ttt{x2} is available for connection only when \ttt{x1} is already modified. 

\section{Some runs}
We try first a coarse simulation of $100,000$ events (10 times the number of agents in the initial system).
\ttt{
\ITE{
\item[\$] KaSim -i ABC.ka -e 100000 -p 1000 -o abc.out
}}
Plotting the content of the \ttt{abc.out} file one notices that nothing of significant interest happen to the observables after 250s. So we can now specify a meaningful time limit by running
\ttt{
\ITE{
\item[\$] KaSim -i ABC.ka -e 100000 -t 250 -p 1000 -o abc.out
}}
which produces the data points whose rendering is given in Fig.~\ref{fig:abc}.
\begin{figure}[htbp]
\begin{center}
\includegraphics[width=12cm]{img/plot-abc.png}
\caption{Simulation of the ABC model: population of unmodified \ttt{C}s (observable \ttt{Cuu} in red) drops rapidly and is replaced, in a first step by simply modified \ttt{C}s (observable \ttt{Cpu} in blue) which are in turn replaced by doubly modified \ttt{C}s (observable \ttt{Cpp} in red). Note that the population of \ttt{AB} complexes (observable \ttt{AB} in black) stabilizes slightly below 400 individuals after about 20s.}
\label{fig:abc}
\end{center}
\end{figure}
We will use this model as a running example for the next chapter, in order to illustrate various advanced concepts.

\chapter{Advanced concepts}\label{chap:advanced}

\section{Perturbation language}\label{sec:mod}

It is possible to use variables\index{variable} of the model as precondition for triggering a \emph{perturbation}\index{perturbation} of the simulation. Note that, by default, a perturbation is applied whenever its pre-condition is satisfied and then discarded. Such perturbation is called \emph{"one shot"}\index{perturbation!one shot}. It is however possible to re-apply the same perturbation each time its pre-condition is satisfied and until a certain condition is met, using the \ttt{repeat ... until} constructors.

Basic perturbations are obtained using the declaration \index{declaration}:
\ITE{
\item[] \ttt{\%mod:} \textit{boolean\_expression} \ttt{do} \textit{effect\_list} 
}
and may be applied repeatedly using:
\ITE{
\item[] \ttt{\%mod:} \ttt{repeat} \textit{boolean\_expression} \ttt{do} \textit{effect\_list} \ttt{until} \textit{boolean\_expression} 
}
where \textit{boolean\_expression} and \textit{effect\_list}\index{effect} are defined by the grammar given  Table~\ref{tab:pert} (the operator \ttt{rel} can be any usual binary relation in $\set{<,=,>}$ and algebraic expressions\index{algebraic expression} are defined Table~\ref{tab:alg}).

\begin{table}[htbp]
  \centering
  \caption{Perturbation expressions}
  \begin{tabular}{@{} lcl @{}}
  	\textit{perturbation\_expression} &::=& \tcb{\ttt{\%mod:}} \textit{perturbation} \\
	&& | \tcb{\ttt{\%mod:}} \tcb{\ttt{repeat}} \textit{perturbation} \tcb{\ttt{until}} \textit{boolean\_expression} \\\\ 
    
    \textit{perturbation} &::=& \textit{boolean\_expression} \tcb{\ttt{do}} \textit{effect\_list} \\\\
    \textit{boolean\_expression} &::=& \textit{algebraic\_expression}\index{boolean expression} \tcb{rel} \textit{algebraic\_expression} \\
    && | (\textit{boolean\_expression} \tcb{||} \textit{boolean\_expression}) \\
    && | (\textit{boolean\_expression} \tcb{\&\&} \textit{boolean\_expression}) \\
    && | \tcb{[not]} \textit{boolean\_expression} \\
    && | \tcb{[true]} | \tcb{[false]} \\\\
    \textit{effect\_list} &::=& \textit{effect} \tcb{\ttt {;}} \textit{effect\_list} | \textit{effect} \\\\
\textit{effect} &::=& \tcb{\$ADD} \textit{algebraic\_expression agent\_expression} \\
    && | \tcb{\$DEL} \textit{algebraic\_expression agent\_expression} \\
    && | \tcb{t <-} \textit{algebraic\_expression} \\
    && | \tcb{\$SNAPSHOT} \textit{opt\_filename}\\ 
    && | \tcb{\$STOP} \textit{opt\_filename}\\
	&& | \tcb{\$FLUX} \textit{opt\_filename} \textit{boolean} \\
    && | \tcb{\$TRACK \var{var\_name}} \textit{boolean} \\
    && | \tcb{\$UPDATE \var{var\_name}} \textit{algebraic\_expression} \\
    && | \tcb{\$PRINT} \tcb{<}\textit{string\_expression}\tcb{>} \\
    && | \tcb{\$PRINTF} \textit{filename} \tcb{<}\textit{string\_expression}\tcb{>} \\
    \\
    \textit{filename} &::=& \tcb{<}\textit{string\_expression}\tcb{>} | \tcb{"filename"} \\
    \textit{opt\_filename} &::=& \textit{filename} | \tcb{$\varepsilon$} \\\\
	\textit{string\_expression} &::=& \tcb{$\varepsilon$} | \tcb{"string" . } \textit{string\_expression} \\
    && | \textit{algebraic\_expression}\tcb{ . } \textit{string\_expression} \\\\
	\textit{boolean} &::=& \tcb{[true]} | \tcb{[false]}
  \end{tabular}
  \label{tab:pert}
\end{table}

The boolean expression is used as a \textit{precondition}\index{precondition} that determines when the perturbation will be triggered, for instance a user writes
\ITE{
\item[] \ttt{\%mod: ([T]>10) \&\& (\var{v1} /\var{v2}) > 1 do $\dots$}
}
to indicate she wishes to trigger a perturbation whenever the simulation time has passed 10 time units and the ratio of variables \ttt{v1} over \ttt{v2} is above 1. Recall that the perturbations are "one shot" interventions on the simulation. Possible interventions are described in the following sections using examples.

\textbf{Note on time dependent preconditions:}
\\Consider a perturbation of the form:
\ITE{
\item[] \ttt{\%mod: {\it f}(t)={\it x} do $\dots$}
}
where $f(t)$ is an algebraic expression dependent on time and $x$ an arbitrary algebraic expression. Having in mind the simulation algorithm implemented by \KaSim, at the beginning of an event loop, both $f(t)$ and $x$ will be evaluated. It is very unlikely (in general with a probability equal to 0) that both values coincide. Currently \KaSim~is not equipped with a solver able to detect that in the past of the current state, there was a $t$ that made the precondition hold, unless the equation is trivial to solve. Therefore the only time dependent precondition with an equality test that is allowed in \KaSim~has to be of the form $\mathtt{[T]=}n$ with $n\in\Nat$. For instance:
\ITE{
\item[] \ttt{\%mod: [T]=10 do \$STOP}
}
will interrupt the simulation after exactly 10 time units.

 
\subsection{Adding or deleting agents during a simulation}

Continuing with the ABC model, the perturbation effect\index{effect}: 
\ttt{\ITE{
\item[] \$ADD $n$ C(x1\int p)
}}
will add $n\geq 0$ instances of \ttt{C} with \ttt{x1} already in state \ttt{p} (and the rest of its interface in the default state as specified line 4 of ABC.ka). Also the perturbation effect\index{effect}:
\ttt{\ITE{
\item[] \$DEL  [inf] B(x!\_)
}}
will remove \emph{all} \ttt{B}s connected to some agent from the mixture\index{mixture}. 

There are various ways one can use perturbations to study more deeply a given kappa model. A basic illustration is the use of a simple perturbation to let a system equilibrate before starting a real simulation. For instance, as can be seen from the curve given in Fig.~\ref{fig:abc}, the number of AB complexes is arbitrarily set to 0 in the initial state (all \ttt{A}s are disconnected from \ttt{B}s in the initial mixture\index{mixture}). In order to avoid this, one can modify the kappa file the following way: we set the initial concentration of \ttt{C} to 0 by deleting line 22. Now we introduce \ttt{C}s after 25 t.u using the perturbation:
\ttt{\ITE{
\item[] \%mod:  [T]=25 do \$ADD 10000 C() 
}}
The modified kappa file is available in the source repository, in the \ttt{model/} directory (file \ttt{abc-pert.ka}). 
Running again a simulation (a bit longer) by entering in the command line:
\ttt{
\ITE{
\item[\$] KaSim -i ABC-pert.ka -e 100000 -t 300 -o abc2.out
}}
one obtains the curve given in Fig.~\ref{fig:abc2}. 

\begin{figure}[htbp]
\begin{center}
\includegraphics[width=10cm]{img/plot-abc2.png}
\caption{Simulation of the ABC model with a perturbation: for t<25s, only \ttt{\textquotesingle a.b\textquotesingle } and \ttt{\textquotesingle a..b\textquotesingle } rules may apply. This enables the concentration of \ttt{\textquotesingle AB\textquotesingle } complexes to go to steady state, before introducing fresh \ttt{C}s at t=25s.}
\label{fig:abc2}
\end{center}
\end{figure}

\subsection{Using snapshots to define a new initial state} 

In the previous example, we let the system evolve for some time without its main reactant \ttt{C} in order to let other reactants go to a less arbitrary initial state. One may object that this way of proceeding is CPU-time consuming if one has to do this at each simulation. An alternative is to use the \ttt{\$SNAPSHOT} primitive that allows a user to export a snapshot of the mixture\index{mixture} at a given time point as a new (piece of) kappa file.
For instance, the declaration\index{declaration}:
\ttt{\ITE{
\item[] \%mod:  [E-]/[E]>0.9 do \$SNAPSHOT \textquotedbl prefix\textquotedbl
}}

will ask \KaSim~to export the mixture\index{mixture} the first time the percentage of null events reaches 90\%. The exported file will be named \ttt{prefix\_$n$.ka} where $n$ is the event number at which the snapshot was taken. One may also use a \textit{string\_expression} to construct any prefix using local variables. Note that one may omit to define a prefix and simply type:
\ttt{\ITE{
\item[] \%mod:  [E-]/[E]>0.9 do \$SNAPSHOT
}}
in which case the default prefix \ttt{snap} will be used for naming snapshots.

Note that if the name already exists a counter will be appended at the end of the file to prevent overwriting.
Note that snapshots can be performed multiple times, for instance every 1000 events, using the declaration\index{declaration}:
\ttt{\ITE{
\item[] \%mod: \ttt{repeat} ([E+] [mod] 1000)=0 do \$SNAPSHOT \textquotedbl abc\textquotedbl~until [false] 
}}
which results in \KaSim~producing a snapshot every 1000 (productive) events until the simulation ends. The perturbation \ttt{\$STOP "final\_state"} will terminate the simulation whenever its precondition is satisfied and produce a snapshot of the last mixture\index{mixture}. Note that instead of producing kappa files, one may use snapshot perturbations to produce an image of the mixture\index{mixture} in the dot format using the parameter \ttt{"dotSnapshots"} (see Section~\ref{sec:param}).

\subsection{Changing the value of a token}
The concentration of any token can be reset on the fly using a perturbation. For instance the declaration\index{declaration}:
\ttt{\ITE{
\item [] \%mod: repeat (|a|<100 do a <- |a|*2) until [false] 
}}
will double the concentration of token \ttt{a} each time it gets below 100.

\subsection{Causality analysis}\label{sec:cflows}

In our ABC example, adding the instruction:
\ttt{\ITE{
\item[] \%mod:  [true] do \$TRACK \var{Cpp} [true]  
}}
\noi will ask~\KaSim~to turn on causality analysis\index{causality} for the observable \ttt{\textquotesingle Cpp\textquotesingle } since the beginning of the simulation, and display the causal explanation of every new occurrence of \ttt{\textquotesingle Cpp\textquotesingle }, until the end of the simulation. The explanation, that we call a \emph{causal flow}\index{causal flow}, is a set of rule application ordered by causality and displayed as a graph using dot format. In this graph, an edge \ttt{r}$\longrightarrow$ \ttt{r\textquotesingle } between two rule applications $\ttt{r}$ and $\ttt{\textquotesingle r\textquotesingle }$ indicates that the first rule application has used, in the simulation, some sites that were modified by the application of the former. We show Fig.~\ref{fig:cflow} an example of such causal flow.

\begin{figure}[htbp]
\begin{center}
\includegraphics[width=5cm]{img/cflow.png}
\caption{Causal flow for the observable \ttt{\textquotesingle Cpp\textquotesingle } of the ABC model. Plain arrows represent causal dependency, dotted arrows show asymmetric conflict between rule occurrences. Here the \ttt{\textquotesingle ab.c\textquotesingle } rule has to occur before the \ttt{\textquotesingle a.b\textquotesingle } rule. Red observable indicate that the last rule allowed one to observe a new instance of \ttt{\textquotesingle Cpp\textquotesingle }.}
\label{fig:cflow}
\end{center}
\end{figure}

Causality analysis of the observable \ttt{Cpp} can be turned off at any time using a declaration\index{declaration} of the form:
\ttt{\ITE{
\item[] \%mod:  [T]>25 do \$TRACK \var{Cpp} [false]  
}}

Each time \KaSim~detects a new occurrence of the observable that is being tracked, it will dump it's causal past as a graph using the dot format (see Fig.~\ref{fig:cflow} above). The name of the file in which the causal flow is stored can be set using the \texttt{\%def} instruction (see Section~\ref{sec:param}). 

\subsubsection*{Compressing causal flows.}

In general pure causal flows\index{causal flow} will contain a lot of information that modelers may not wish to consider. Indeed in classical flows, causality (represented by an edge between to rule applications in the graph) is purely local. Therefore a sequence $a\rar b \rar c$ only implies that an instance of rule $a$ caused an instance of rule $b$ which in turn created an instance of the observable $c$. However it does not imply that $a$ was "necessary" for $c$ to occur (for instance $c$ might have been possible before $a$ but not after, and $b$ would be simply re-enabling $c$). It is possible to tell \KaSim~to retain only events that are more strongly related to the observable using two compression techniques (see Ref.~\cite{Dan_etal12} for formal details). Intuitively, in a \emph{weakly} compressed causal\index{weak compression} flow one has the additional property that if an event $e$ is a (possibly indirect) cause of the observable, then preventing $e$ from occurring would have prevented the rest of the causal flow to occur (\ie it is not possible to reconstruct a computation trace containing the observable with the events that remain in the causal flow). A \emph{strongly} compressed causal flow enjoys the same property with an additional level of compression obtained by considering different instances of the same rule to be indistinguishable\index{strong compression}. Note that causal flow compressions may be memory and computation demanding. For large systems it may be safer to start with weak compressions only. 

The type of compression can be set using the \texttt{\%def} instruction (see Section~\ref{sec:param}). For instance:
\ttt{\ITE{
\item [] \%def: "displayCompression" "none" "weak" "strong"
}}
will ask \KaSim~to output 3 versions of each computed causal flow, with all possible degrees of compressions. Each causal flow is outputted into a file \texttt{[filename][Type]\_$n$.dot} where \texttt{filename} is the default name for causal flows which can be redefined using the parameter \texttt{cflowFileName}, \texttt{Type} is the type of compression (either nothing or \texttt{Strongly}, or \texttt{Weakly}) and $n$ is the identifier of the causal flow. For each compression type a summary file, named \texttt{[filename][Type]Summary.dat}, is also produced. It allows to map each compressed causal flow to the identifier of its uncompressed version (row \texttt{\#id}), together with the production time $T$ and event number $E$ at which the observable was produced. It also contains information about the size of the causal flow.

\subsection{Flux maps}

\begin{figure}[h] %  figure placement: here, top, bottom, or page
   \centering
   \includegraphics[width=6cm]{img/flux.pdf} 
   \caption{Flux map of the \ttt{abc.ka} model, taken from t=0 to t=20 time units. The \ttt{A} releasing rules \ttt{a..b} and \ttt{mod x2} are contributing very little to the activity of \ttt{a.c} which is a sign of an excess of free \ttt{A}s in the system at this time interval.}
   \label{fig:flux}
\end{figure}
%%
The \emph{flux map}\index{flux map} is a powerful observation that tracks, on the fly, the influence that rule applications have on each others. It is dynamically generated and tracks effective impacts (positive or negative) a every rule application. The flux map can be computed using declaration\index{declaration}s of the form:
\ttt{\ITE{
\item [] \%mod: [true] do \$FLUX "flux.dot" [true]
\item [] \%mod: [T]>20 do \$FLUX "flux.dot" [false]
}}
The resulting \emph{flux map} is a graph where a positive edge between rules $r$ and $s$ (in green) indicates an overall positive contribution of $r$ over $s$. Said otherwise, the sum of $r$ applications increased the activity\index{activity} of $s$. Conversely, a negative edge (in red) will indicate that $r$ had an overall negative impact on the activity\index{activity} of $s$. Note that the importance of the flux between two rules can be observed by looking at the label on the edges that indicate the overall activity transfer (positive or negative) between the rules. The above declaration\index{declaration} produce a flux map that is shown Fig.~\ref{fig:flux}. Note that flux may vary during time, therefore the time or event limit of the simulation is of importance and will likely change the aspect of the produced map.


\subsection{Updating kinetic rates on the fly}

Any variable between simple quotes can be updated during a simulation using a declaration\index{declaration} of the form:
\ttt{\ITE{
\item [] \%mod: \var{Cpp}~> 500 do \$UPDATE \var{k\_on} 0.0
}}
This perturbation will be applied whenever the observable \ttt{\var{Cpp}} will become greater than 500. Its effect will be to set the on rate of all binding rules to 0. Note that according to the grammar given Table~\ref{tab:pert}, one may use any algebraic expression\index{algebraic expression} as the new value of the variable. For instance:
\ttt{\ITE{
\item [] \%mod: \var{Cpp}~> 500 do \$UPDATE \var{k\_on} \var{k\_on}/100
}}
\noi will cause the on rate of all rules to decrease a hunderd fold. Note that it is possible to override the kinetic rate of a specific rule:  in our ABC example, the declaration\index{declaration}:
\ttt{\ITE{
\item [] \%mod: \var{Cpp}~> 500 do \$UPDATE \var{a.b} [inf]
}}
\noi will set the kinetic rate of rule \ttt{\var{a.b}}~to infinity. 

\subsection{Combining several effects in a single perturbation}
As an example, consider the computation of causal flows between $t=10$ and $t=20$ using the declarations:
 \ttt{\ITE{
\item [] \%mod: [T]>10 do \$TRACK \var{Cpp} [true]
\item [] \%mod: [T]>20 do \$TRACK \var{Cpp} [false]
}}

The above declaration will ask \KaSim~to analyze each new occurrence of \var{Cpp} in that time interval. If $n$ new instances took place, then \KaSim~will have to compute $n$ causal flows. One may want to bound the number of computed flows to a certain value, say 10. One may do so using the combination of perturbations and variables given below:
 \ttt{\ITE{
\item[] \%var: \var{x} 0
\item[] \%mod: [T]>10 do (\$TRACK \var{Cpp} [true] ; \$UPDATE \var{x} \var{Cpp})
\item[] \%mod: [T]>20 || (\var{x} > 0 \&\& \var{Cpp} - \var{x} > 9) do \$TRACK \var{Cpp} [false]
}}
The first line is a declaration of an $x$ variable that is initially set to 0. Note that the second line is a perturbation that contains two simultaneous effects, the first one triggering causality analysis and the second one updating the value of variable $x$ to the current value of variable \ttt{\var{Cpp}}.
The last line stops causality analysis whenever time is greater than 20 or when 10 new observables have been found (the difference between the current value of \ttt{\var{Cpp}} and $x$.

\subsection{Printing values during a simulation}
The effect \ttt{\$PRINT <\textit{string\_expression}>} enables one to output values during a computation to standard output, or to a specific file when using \ttt{\$PRINTF}. For instance:
 \ttt{\ITE{
\item[] \%mod: repeat \textbackslash
\item[] |A|<0 do \$PRINTF <"token\_".[E+].".dat" <"Token A is: " . |A| . " at time=". [T]>\textbackslash 
\item [] until [false]
}}
will ask \KaSim~to output the value of token \ttt{A} in a file "token\_$n$.dat" which changes at each new productive event, each time its value gets below 0.

\section{Link type}

In standard kappa, in order to require a site to be bound for an interaction to occur, one may use the \emph{semi-link}\index{semi-link} construct \ttt{!\_} which does not specify who the partner of the bond is. For instance in the variable:
\ttt{\ITE{
\item[] \%var: \var{ab}~A(x!\_),B(y!\_)
}}
will count the number of \ttt{A}s and \ttt{B}s connected to someone, including the limit case \ttt{A(x!1),B(y!1)}. It is sometimes convenient to specify the \emph{type}\index{link type} of the semi-link, in order to restrict the choice of the binding partner. For instance the variable:
\ttt{\ITE{
\item[] \%var: \var{ab}~A(x!y.B),B(y!x.A)
}}
will count the number of \ttt{A}s whose site \ttt{x} is connected to a site \ttt{y} of \ttt{B}, plus the number of \ttt{B}s whose site \ttt{y} is connected to a site \ttt{x} of \ttt{A}. Note that this still includes the case \ttt{A(x!1),B(y!1)}.


\section{Implicit signature}
\KaSim~permits users in a hurry to avoid writing agent signatures explicitly using the option \ttt{-\,-implicit-signature} of the command line. The signature is then deduced using information gathered in the KF\index{kappa file}. Note that it is not recommended to use the DCDW convention for introduced agents in conjunction with the \ttt{-\,-implicit-signature} option unless the default state of all sites is mentioned in the \ttt{\%init} declaration\index{declaration}s or in the rules that create agents. 

\section{Simulation packages}

The simulation algorithm that is implemented in \KaSim~requires an initialization phase whose complexity is proportional to $R*G$ where $R$ is the cardinal of the rule set and $G$ the size of the initial mixture\index{mixture}. Thus for large systems, initialization may take a while. Whenever a user wishes to run several simulations of the \emph{same} kappa model, it is possible to skip this initialization phase by creating a \emph{simulation package}\index{simulation package}. For instance:
\ttt{\ITE{
\item [] KaSim -i abc.ka -t $n$ -make-sim abc.kasim
}}
will generate a standard simulation of the \ttt{abc.ka} model, but in addition, will create the simulation package \ttt{abc.kasim} (.kasim extension is not mandatory). This package is a binary file, \ie not human readable, that can be used as input of a new simulation using the command:
\ttt{\ITE{
\item [] KaSim -load-sim abc.kasim -t $k$ 
}}
Note that this simulation is now run for $k$ time units instead of $n$. Importantly, simulation packages can only be given as input to the \emph{same} \KaSim~that produced it. As a consequence, recompiling the code, or obtaining different binaries, will cause the simulation package to become useless.

\section{Simulation parameters configuration}\label{sec:param}

In the KF\index{kappa file} (usually in a dedicated file) one may use expressions of the form:
\[
\hbox{\ttt{\%def: "}\textit{parameter\_name}\ttt{" "}\textit{parameter\_value}\ttt{"}}
\]
where tunable parameters are described table~\ref{tab:parameters} (default values are given first in the possible values column).
%%
\begin{table}[ht]
\caption{User defined parameters}
\begin{tabular}{l|l|lll}
parameter & possible values & description \\
\hline &\\
\textit{Causality analysis}&\\
\ttt{"displayCompression"} & any combination of \\
&  \ttt{"none", "strong", "weak"} & \small type of compression \\
\ttt{"cflowFileName"} & \ttt{"cflow.dot"}, any string & \small file name for causal flows\\ & \\
\ttt{"showIntroEvents"} & \ttt{"no", "yes"} & \small show introduction events in causal flows\\ & \\
\textit{Pretty printing}&\\
\ttt{"plotSepChar"} & \ttt{" "} or any character & \small separation character for plots\\
\ttt{"dotSnapshots"} & \ttt{"no", "yes"} & \small generate dot snapshots\\
\ttt{"colorDot"} &  \ttt{"no", "yes"} & \small use colors in dot format files\\
\ttt{"progressBarSymbol"} & \ttt{"\#"} or any character & \small symbol for the progress bar\\
\ttt{"progressBarSize"} & \ttt{"60"} or any integer & \small length of the progress bar\\ &&\\
\textit{Simulation options} &\\
\ttt{"dumpIfDeadlocked"} & \ttt{"no","yes"} & \small Snapshot when simulation \\&&\small is stalled\\
\ttt{"maxConsecutiveClash"} & \ttt{"2"} or any integer & \small number of consecutive clashes \\ && \small before giving up \\ && \small square approximation\\
\textit{Miscellaneous} &\\
\ttt{"dumpInfluenceMap"} & \ttt{"no","yes"} & \small Output the influence map \\ &&\small of the model\\
\ttt{"influenceMapFileName"} & \ttt{"im.dot", any string} & \small file name for the influence map\\

\end{tabular}
\label{tab:parameters}
\end{table}
%%

\subsection{The Influence map}
The influence map of a model is an object that may help modelers checking the consistency of the rule set they use. It is generated if the modeler has required it using the parameter \ttt{"dumpInfluenceMap"} which will generate the so called \emph{influence map}\index{influence map} of the model.

Unlike the flux map, the influence map is \emph{statically} computed and does not depend on kinetic rates nor initial conditions. It describes how rules \emph{with no side effect}\index{side effect} may potentially influence each other during a simulation. \KaSim~will produce a dot format file containing the influence relation over all rules and observables of the model. The produced graph visualized using a circular rendering\footnote{One may use for instance the \ttt{circo} program that is part of the \textit{graphviz} suite.} is given in Figure~\ref{fig:abc-im}. Observables are represented as circular nodes\footnote{Observable can be viewed as identity rules for computing influence.} and rules as rectangular nodes. Edges are decorated with the list of embeddings (separated by a semi-colon) allowing rules\textquotesingle s right hand sides to be mapped to left hand sides. Note that the influence map will not display relations between rules that are induced by side effects (see Section~\ref{sec:side-effect}).

\begin{figure}[htbp] %  figure placement: here, top, bottom, or page
   \centering
   \includegraphics[width=15cm]{img/abc-im.png} 
   \caption{The influence map of the \ttt{abc.ka} model defined in Chapter~\ref{chap:abc}. Note that observable \ttt{C(x1\int u?,x2\int u?)} is not activated by any other rule. This implies that its number of occurrences does not change during a simulation. Edge labels denote embeddings with the convention that $[i\rar j]$ denotes the embedding of agent number $i$ of the origin\textquotesingle s rhs, into agent $j$ of the target\textquotesingle s lhs.}
   \label{fig:abc-im}
\end{figure}

More formally, consider the rules $r:L\rar R$ and $s:L'\rar R'$. One wishes to know whether it is possible that the application of rule $r$ over a graph $G$ creates a new instance of rule $s$. To do so, \KaSim~will try to generate a term $T$ containing a match for $R$ and $L'$ that overlap on some sites that are modified by $r$ (see Figure~\ref{fig:im} for illustration).

\begin{figure}[htbp] %  figure placement: here, top, bottom, or page
   \centering
   \includegraphics[width=8cm]{img/im.png} 
   \caption{Computation of the influence of the top rule on the rule below: the right hand side of the first rules embeds in a common term with the left hand side of the second rule. It results that the first rule has a positive influence on the second.}
   \label{fig:im}
\end{figure}

\chapter{Frequently asked questions}
\section*{Simulation hangs after a while}
If the progress bar seems stalled, it does not necessarily mean that the simulation is blocked. In particular when a simulation is triggered with a \emph{time} limit (\ttt{-t} option of the command line) it might only indicate that the bio clock is stalled while computation events still occur. Recall that the average (bio) time one has to wait in order to apply a rule is $1/A$, where $A$ is the sum of all the rule activities (which is equal to the number of instances that a rule has, times its kinetic rate). Whenever the number of occurrences of a rule grows too fast (if new agents are created during the simulation for instance), or if the kinetic rate of a rule is defined by a function that grows rapidly, the average time increment might tend to 0 and if it remains so for a while, it will block the progress bar whose advance is proportional to the bio time \ttt{[T]}. 

In order to make sure that \KaSim~is not incorrectly blocked you may wish to plot the event clock against time clock using the observable \ttt{\%obs: \textquotesingle events\textquotesingle  [E]} or run the simulation using an event limit (\ttt{-e} option of the command line) instead of a time limit.

\section*{What do null events mean, why do I have any?}

Null events\index{null event} is a way for \KaSim~to compensate for some over approximation it is doing, in order deal with large simulations more efficiently. They usually do not impact significantly the performances of the simulator, unless the model contains rules using the special notation to deal with ambiguous molecularity (see Section~\ref{sec:ambiguous}). With pure Kappa rules, the ratio $r$ of null event over productive ones (that you can track using the observable \ttt{\%obs: \textquotesingle r\textquotesingle   [E-]/[E+]}) should tend to 0 when models have a lot of agents.

\section*{No data points are generated}
Make sure you have \ttt{\%obs} or \ttt{\%plot} instructions in your KF\index{kappa file}. Also make sure to use the \ttt{-p} option in the command line to tell KaSim how many points you wish to have on your curves. 

\section*{Too many instances of an observable}
The value of a kappa expression $E$  is equal to the number of embeddings\index{embedding} it has in the current mixture\index{mixture} $M$. Embeddings are maps from agents in $E$  to agents in $M$. If $E$ has symmetries then every permutation of $E$ will be counted as a new embedding. For instance let $E=$\ttt{A(x!1),A(x!1)} and let $M=$\ttt{A(x!1,y\int p),A(x!1,y\int u)}. \KaSim~will count two instances of $E$ in $M$: the one mapping the first \ttt{A} of $E$ to the first \ttt{A} of $M$ and the one mapping the first \ttt{A} of $E$ to the second \ttt{A} of $M$.
 
\section*{The computed influence map is incorrect, it misses some activation or has to much of them}
The influence map\index{influence map} contains relations that are computed on side effect free rules only. It is likely that a missing activation is due to a side effect that is not taken into account. If the influence map shows an activation between rule $r$ and $s$ that is never possible with a given model, just remember that activation computation implies that \emph{there exists} a context in which applying rule $r$ will create a new instance of rule $s$. This context might simply never be realized with the given rules or initial conditions.

\section*{Value \ttt{nan} in the data file\index{data file} at the end of the simulation}
The value \ttt{nan} means "Not a Number". It is generated when a plotted variable is infinite. Make sure this variable is not divided by zero at some point.

\bibliographystyle{plain}
\bibliography{fmb,concur} 

\printindex

\end{document}  
